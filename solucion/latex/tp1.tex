\documentclass[10pt, a4paper]{article}
\usepackage[paper=a4paper, left=1.5cm, right=1.5cm, bottom=1.5cm, top=3.5cm]{geometry}
\usepackage[latin1]{inputenc}
\usepackage[T1]{fontenc}
\usepackage[spanish]{babel}
\usepackage{indentfirst}
\usepackage{fancyhdr}
\usepackage{latexsym}
\usepackage{lastpage}
\usepackage{aed2-symb,aed2-itef,aed2-tad,caratula}
\usepackage[colorlinks=true, linkcolor=blue]{hyperref}
\usepackage{calc}

\newcommand{\f}[1]{\text{#1}}
\renewcommand{\paratodo}[2]{$\forall~#2$: #1}
\newcommand{\ifCustom}[3]{\textbf{if} #1 \\ \textbf{then} #2 \\ \textbf{else} #3 \\ \textbf{fi}}
\newcommand{\ifInline}[3]{\textbf{if} #1 \textbf{then} #2 \textbf{else} #3 \textbf{fi}}
\newcommand{\tupla}[1]{$\langle$ #1 $\rangle$}
\sloppy

\hypersetup{%
 % Para que el PDF se abra a pagina completa.
 pdfstartview= {FitH \hypercalcbp{\paperheight-\topmargin-1in-\headheight}},
 pdfauthor={C�tedra de Algoritmos y Estructuras de Datos II - DC - UBA},
 pdfkeywords={TADs basicos},
 pdfsubject={Tipos abstractos de datos basicos}
}

\parskip=5pt % 10pt es el tama�o de fuente

% Pongo en 0 la distancia extra entre �temes.
\let\olditemize\itemize
\def\itemize{\olditemize\itemsep=0pt}

% Acomodo fancyhdr.
\pagestyle{fancy}
\thispagestyle{fancy}
\addtolength{\headheight}{1pt}
\lhead{Algoritmos y Estructuras de Datos II}
\rhead{$1^{\mathrm{er}}$ cuatrimestre de 2019}
\cfoot{\thepage /\pageref{LastPage}}
\renewcommand{\footrulewidth}{0.4pt}

\author{Algoritmos y Estructuras de Datos II, DC, UBA.}
\date{}
\title{Tipos abstractos de datos b�sicos}

\begin{document}

\materia{Algoritmos y Estructura de Datos 2}
\titulo{Exorcismo Extremo}
\subtitulo{TP1}

\integrante{Rosinov, Gaston Einan}{37/18}{grosinov@gmail.com}
\integrante{Schuster, Martin Ariel}{208/18}{m.a.schuster98@gmail.com}
\integrante{Panichelli, Manuel}{72/18}{panicmanu@gmail.com}

%Pagina de titulo e indice
\thispagestyle{empty}

\maketitle
\tableofcontents

\newpage

% TADs

\section{\tadNombre{Extensiones y Renombres}}

\textbf{TAD} \tadNombre{Fantasma} \textbf{ES} \tadNombre{Nat} \\ \\
\textbf{TAD} \tadNombre{PJ} \textbf{ES} \tadNombre{Nat} \\ \\
\textbf{TAD} \tadNombre{Posicion} \textbf{ES} \tadNombre{Tupla(Nat, Nat)}

\begin{tad}{\tadNombre{Nat} \textbf{extiende} \tadNombre{Nat}}

\tadOtrasOperaciones
    \tadOperacion{$\bullet \ \% \ \bullet$}{nat, nat}{nat}{}
    \tadAxiomas[\paratodo{nat}{n,m}]
    \tadAxioma{n $\%$ m}{\textbf{if} n < \ m \textbf{then} n \textbf{else} (n - m) $\%$ m \textbf{fi}}
\end{tad}

\begin{tad}{\tadNombre{Ubicacion} \textbf{extiende} \tadNombre{Tupla(Posicion, Direccion)}}
    \tadOtrasOperaciones
        \tadAlinearFunciones{pos}{ubicacion}{direccion}
        \tadOperacion{pos}{ubicacion}{posicion}{}
        \tadOperacion{dir}{ubicacion}{direccion}{}

    \tadAxiomas[\paratodo{ubicacion}{u}]
        \tadAlinearAxiomas{pos(u)}
        \tadAxioma{pos(u)}{$\Pi_1$(u)}
        \tadAxioma{dir(u)}{$\Pi_2$(u)}
\end{tad}

\begin{tad}{\tadNombre{Secuencia} \textbf{extiende} \tadNombre{Secuencia}}
\tadOtrasOperaciones
    \tadOperacion{$\bullet[\bullet]$}{secu($\alpha$)/s, nat/i}{$\alpha$}{i < \ long(s)}
    \tadAxiomas[\paratodo{secu($\alpha$)}{s}, \paratodo{nat}{i}]
    \tadAxioma{s$[i]$}{\textbf{if} $i = 0?$ \textbf{then} prim(s) \textbf{else} fin(s)$[i - 1]$ \textbf{fi}}
\end{tad}


\newpage

\section{TAD \tadNombre{Juego}}
\begin{tad}{\tadNombre{Juego}}
    \tadGeneros{juego}
    \tadExporta{juego, observadores, generadores, puntaje, ronda}
    \tadUsa{\tadNombre{Secuencia, Nat, Conjunto, Habitacion,  Ubicacion, \\
            PJ, Accion, Direccion, Fantasma, PJ, Diccionario, Bool}}
    \tadIgualdadObservacional{j}{j'}{juego}{
        (accionesPJs(j) $\igobs$ accionesPJs(j')) $\land$ \\
        (accionesFan(j) $\igobs$ accionesFan(j')) $\land$ \\
        (localizarJugadores(j) $\igobs$ localizarJugadores(j')) $\land$ \\
        (hab(j) $\igobs$ hab(j')) $\land$ \\
        (($\forall$ p : pj) (vivePJ?(j, p) \igobs vivePJ?(j', p))) $\land$ \\
        (($\forall$ f : fantasma) ((viveFan?(j, p) \igobs viveFan?(j', p)) $\land$ \\
                                   (ubicacionInicialFan(j, f) \igobs ubicacionInicialFan(j', f)))
    }

    
    \tadAlinearFunciones{nombreSiguienteFan}{conj(pj)/pjs, secu(accion)/as, ubicacion/u}

    \tadObservadores
        \tadOperacion{accionesPJs}{juego}{dicc(pj, secu(accion))}{}
        \tadOperacion{accionesFan}{juego}{dicc(pj, secu(accion))}{}
        \tadOperacion{hab}{juego}{hab}{}
        \tadOperacion{vivePJ?}{juego/j, pj/p}{bool}{p $\in$ jugadores(j)}
        \tadOperacion{viveFan?}{juego/j, fantasma/f}{bool}{f $\in$ fantasmas(j)}
        \tadOperacion{ubicacionInicialFan}{juego/j, fantasma/f}{ubicacion}{f $\in$ fantasmas(f)}
        \tadOperacion{\textbf{localizarJugadores}}{juego}{dicc(pj, ubicacion)}{}
    
    \tadGeneradores
        \tadOperacion{iniciar}
            {conj(pj)/pjs, secu(accion)/as, ubicacion/u, hab/h}
            {juego}
            {esConexa?(h) $\land$ \\
             $\neg$ $\emptyset$?(as) $\land$ \\
             $\neg$ $\emptyset$?(pjs) $\land$ \\
             esValida?(h, pos(u))}

        \tadOperacion{proxPaso}
            {juego/j, pj/p, accion/a}
            {juego}
            {p $\in$ jugadores(j) $\yluego$ \\
             vivePJ?(j, p) $\land$ \\
             $\neg$ termino?(j) $\land$ \\
             $\neg$ esMirar(a)}
    \tadOtrasOperaciones
        \tadOperacion{jugadores}{juego}{conj(pj)}{}
        \tadOperacion{fantasmas}{juego}{conj(fantasma)}{}
        \tadOperacion{nombreSiguienteFan}{juego}{fantasma}{}
        \tadOperacion{puntaje}{juego}{nat}{}
        \tadOperacion{ronda}{juego}{nat}{}
        \tadOperacion{paso}{juego}{nat}{}
        \tadOperacion{cantAcciones}{juego, conj(pj)}{nat}{}
        \tadOperacion{terminaRonda}{juego/j, pj/p, accion}{bool}{p $\in$ jugadores(j)}
        \tadOperacion{fantasmaEspecial}{juego}{fantasma}{}
        \tadOperacion{termino?}{juego}{bool}{}
        \tadOperacion{estanVivos}{juego/, conj(pj)/pjs}{bool}{pjs $\subseteq$ jugadores(j)}
        \tadOperacion{ubicacionInicialPJ}{juego/j, pj/p}{ubicacion}{p $\in$ jugadores(j)}
        \tadOperacion{ubicacionPJ}{juego/j, pj/p}{ubicacion}{p $\in$ jugadores(j)}
        \tadOperacion{ubicacionFan}{juego/j, fantamsa/f}{ubicacion}{f $\in$ fantasmas(j)}
        \tadOperacion{deducirUbicacion}{juego/j, ubicacion/u, acciones}{ubicacion}{esValida?(hab(j), pos(u))}
        \tadOperacion{moriraFantasma}{juego/j, pj/p, accion, fantasma/f}{bool}
            {p $\in$ jugadores(j) $\land$ \\
             f $\in$ fantasmas(j) $\yluego$ \\
             viveFan?(j, f) $\land$ \\
             vivePJ?(j, p)}
        \tadOperacion{moriraPJ}{juego/j, conj(fantasma)/fs, pj/p, accion}{bool}
            {p $\in$ jugadores(j) $\land$ \\
             fs $\subseteq$ fantasmas(j) $\yluego$ \\
             vivePJ?(j, p)}
        \tadOperacion{moriraPJPorFan}{juego/j, fantasma/f, pj/p, accion}{bool}
            {p $\in$ jugadores(j) $\land$ \\
             f $\in$ fantasmas(j) $\yluego$ \\
             vivePJ?(j, p) $\land$ \\
             viveFan?(j, f)}
        \tadOperacion{accionFan}{juego/j, fantasma/f}{accion}{f $\in$ fantasmas(j) $\yluego$ \\ viveFan?(j, f)}
        \tadOperacion{inicializarAcciones}{conj(pj)}{dicc(pj, secu(accion))}{}
        \tadOperacion{agregarFantasma}
            {juego/j, ubicacion/u, dicc(fantasma, secu(accion)), fantasma, secu(accion)}
            {dicc(fantasma, secu(accion))}
            {esValida?(hab(j), pos(u))}
        \tadOperacion{generarAccionesFan}{juego/j, ubicacion/u, secu(accion)}{secu(accion)}{esValida?(hab(j), pos(u))}

    \tadAxiomas[\paratodo{pj}{p},\\
            \paratodo{conj(pj)}{pjs},\\
            \paratodo{fantasma}{f},\\
            \paratodo{conj(fantasma)}{fs},\\
            \paratodo{juego}{j},\\
            \paratodo{hab}{h},\\
            \paratodo{ubicacion}{u, uInicialPJ},\\
            \paratodo{accion}{a},\\
            \paratodo{secu(accion)}{as}
        ]
        \tadAlinearAxiomas{ubicacionInicialFan(iniciar(pjs, as, u, h))}

        % Observadores
        % ------------

        % accionesPJs
        \tadAxioma{accionesPJs(iniciar(pjs, as, u, h))}{inicializarAcciones(pjs)}
        \tadAxioma{accionesPJs(proxPaso(j, p, a))}{
            \ifCustom
                {$\neg$ terminaRonda(j, p, a)}
                {definir(p, obtener(p, accionesPJs(j)) \circulito a, accionesPJs(j))}
                {inicializarAcciones(jugadores(j))}
        }
        
        % accionesFan
        \tadAxioma{accionesFan(iniciar(pjs, as, u, h))}{definir(nombreSiguienteFan(j), as, vacio)}
        \tadAxioma{accionesFan(proxPaso(j, p, a))}{
            \ifCustom
                {$\neg$ terminaRonda(j, p, a)}
                {accionesFan(j)}
                % Como termina la ronda luego de la acción de p,
                % p mató al fantasma especial
                % Le agrego las acciones de p al nuevo fantasma.
                {agregarFantasma(j,
                    ubicacionInicialPJ(j, p),
                    accionesFan(j),
                    nombreSiguienteFan(j),
                    obtener(p, accionesPJs(j)) \circulito a
                )}
        }
        
        % hab
        \tadAxioma{hab(iniciar(pjs, as, u, h))}{h}
        \tadAxioma{hab(proxPaso(j, p, a))}{hab(j)}

        % vivePJ?
        \tadAxioma{vivePJ?(iniciar(pjs, as, u, h), p')}{true}
        \tadAxioma{vivePJ?(proxPaso(j, p, a), p')}{
            % Si con la acción de p termina la ronda, reviven todos
            terminaRonda?(j, p, a) $\lor$ \\
             \ifCustom
                {p = p'}
                % Entonces por la reestricción de prox paso se que no estaba muerto en j.
                {$\neg$ moriraPJ(j, fantasmas(j), p, a)}
                {vivePJ?(j, p') $\yluego$
                 $\neg$ moriraPJ(j, fantasmas(j), p, a)}
        }

        % viveFan?
        \tadAxioma{viveFan?(iniciar(pjs, as, u, h), f)}{true}
        \tadAxioma{viveFan?(proxPaso(j, p, a), f)}{
            terminaRonda?(j, p, a) $\lor$ \\
             (viveFan?(j, f) $\yluego$ $\neg$ moriraFantasma(j, p, a, f))
        }
        
        % ubicacionInicialFan
        \tadAxioma{ubicacionInicialFan(iniciar(pjs, as, u, h))}{u}
        \tadAxioma{ubicacionInicialFan(proxPaso(j, p, a))}{
            % si f no pertenece a los fantasmas del juego antes de efectuar el paso,
            % entonces es nuevo, y fue la acción de p la que finalizó la ronda,
            % así agregando un nuevo fantasma.
            \ifCustom
                {f $\in$ fantasmas(j)}
                {ubicacionInicialFan(j, f)}
                {ubicacionInicialPJ(j, p)}
        }

        % Otras operaciones
        % -----------------

        \tadAxioma{jugadores(j)}{claves(accionesPJs(j))}
        \tadAxioma{fantasmas(j)}{claves(accionesFan(j))}
        \tadAxioma{nombreSiguienteFan(j)}{\#(claves(accionesFan(j))) + 1}

        \tadAxioma{puntaje(j)}{ronda(j) - 1}
        \tadAxioma{ronda(j)}{\#(fantasmas(j))}
        \tadAxioma{paso(j)}{cantAcciones(j, jugadores(j))}

        \tadAxioma{cantAcciones(j, pjs)}{
            \ifCustom
                {$\emptyset$?(pjs)}
                {0}
                {long(obtener(dameUno(pjs), accionesPJs(j))) +\\
                 cantAcciones(j, sinUno(pjs))}
        }
        
        \tadAxioma{termino?(j)}{$\neg$ estanVivos(j, jugadores(j))}
        \tadAxioma{estanVivos(j, pjs)}{
            \ifCustom
                {$\emptyset$?(pjs)}
                {true}
                {vivePJ?(j, dameUno(pjs)) $\land$ \\
                 estanVivos(j, sinUno(pjs))}
        }

        \tadAxioma{fantasmaEspecial(j)}{\#(claves(accionesFan(j)))}

        \tadAxioma{ubicacionInicialPJ(j, p)}{obtener(p, localizarJugadores(j))}

        \tadAxioma{ubicacionPJ(j, p)}{
            deducirUbicacion(j, ubicacionInicialPJ(j, p), \\
                             obtener(p, accionesPJs(j)))
        }

        \tadAxioma{ubicacionFan(j, f)}{
            deducirUbicacion(j, ubicacionInicialFan(j, f), \\
                             obtener(f, accionesFan(j)))
        }
            
        \tadAxioma{deducirUbicacion(j, u, as)}{
            \ifCustom
                {vacia?(as)}
                {u}
                {deducirUbicacion(j, ubicacionLuegoDe(prim(as), hab(j), u), fin(as))}
        }

        \tadAxioma{agregarFantasma(j, uInicialPJ, \\accionesFantasmas, f, as)}{
            definir(f, generarAccionesFantasma(j, uInicialPJ, as), \\
                    accionesFantasmas)
        }

        \tadAxioma{generarAccionesFan(j, uInicialPJ, as)}{
            as \\
            \& (\secuencia{nada $\bullet$ nada $\bullet$ nada $\bullet$ nada $\bullet$ nada}) \\
            \& invertir(hab(j), uInicialPJ, as)
        }

        \tadAxioma{inicializarAcciones(pjs)}{
            \ifCustom
                {$\emptyset$?(pjs)}
                {vacio}
                {definir(dameUno(pjs), \secuencia{}, \\
                         inicializarAcciones(sinUno(pjs)))}
        }

        \tadAxioma{terminaRonda(j, p, a)}{moriraFantasma(j, p, a, fantasmaEspecial(j))}
        
        \tadAxioma{moriraFantasma(j, p, a, f)}{
            pos(ubicacionFan(j, f)) $\in$ \\
            posicionesAfectadasPor(a, hab(j), ubicacionPJ(j, p))
        }

        \tadAxioma{moriraPJ(j, fs, p, a)}{
            \ifCustom
                {$\emptyset$?(fs)}
                {false}
                {(viveFan?(j, dameUno(fs)) $\yluego$ \\
                 moriraPJPorFan(j, dameUno(fs), p, a)) $\oluego$ \\
                 moriraPJ(j, sinUno(fs), p, a)}
        }

        \tadAxioma{moriraPJPorFan(j, f, p, a)}{
            $\neg$ moriraFantasma(j, p, a, f) $\land$ \\
            (pos(ubicacionLuegoDe(a, hab(j), ubicacionPJ(j, p))) $\in$ \\
             posicionesAfectadasPor(accionFan(j, f), hab(j), \\
                                    ubicacionFan(j, f)))
        }

        \tadAxioma{accionFan(j, f)}{obtener(accionesFan(j), f)[paso(j) \% obtener(accionesFan(j), f)]}
\end{tad}

\section{TAD \tadNombre{Habitacion}}

\begin{tad}{\tadNombre{Habitacion}}
  \tadGeneros{hab}
  \tadExporta{hab, observadores, generadores, esConexa?}
  \tadUsa{\tadNombre{posicion, bool, nat}}

  \tadIgualdadObservacional{h}{h'}{hab}{$(\forall p:\text{posicion})$(esValida?($p, h$) $\igobs$ esValida?($p, h'$) $\yluego$ \\ (esValida?($p, h$) $\impluego$ \\(estaOcupada?($p, h$) $\igobs$ estaOcupada?($p, h'$))))}
  \tadAlinearFunciones{verificarAlcancePos}{hab,conj(posicion),conj(posicion)}


  \tadObservadores
    \tadOperacion{esValida?}{hab,posicion}{bool}{} %Decisión tomada, tiene que ser de 2 o mas porque sino nadie muere.
    \tadOperacion{estaOcupada?}{hab/h,posicion/p}{bool}{esValida?(h, p)}

  \tadGeneradores
    \tadOperacion{nueva}{nat/n}{hab}{n>1}
    \tadOperacion{ocupar}{hab/h,posicion/p}{hab}{esValida?(h, p) \yluego $\neg$ estaOcupada?(h, p)}

  \tadOtrasOperaciones
    \tadOperacion{esConexa?}{hab}{bool}{}
    \tadOperacion{tamano}{hab}{nat}{}
    \tadOperacion{posiciones}{hab}{conj(posicion)}{}
    \tadOperacion{darPosiciones}{hab, nat, nat, nat}{conj(posicion)}{}
    \tadOperacion{posicionesLibres}{hab/h,conj(posicion)/ps}{conj(posicion)}{ps $\subseteq$ posiciones(h)}
    \tadOperacion{verificarAlcance}{hab/h,conj(posicion)/ps}{bool}{ps $\subseteq$ posiciones(h)}
    \tadOperacion{verificarAlcancePos}{hab/h,conj(posicion)/ps,posicion/p}{bool}{ps $\subseteq$ posiciones(h) $\land$ p $\in$ posiciones(h)}
    \tadOperacion{\textbf{esAlcanzable}}{hab/h, pos/p, pos/p'}{bool}
        {esValida?(h, p) $\land$ esValida?(h, p') $\yluego$ \\
         ($\neg$ estaOcupada?(h, p) $\land$ $\neg$ estaOcupada?(h, p))}

\tadAlinearFunciones{estaOcupada?(ocupar(h,p'),p)}{}

  \tadAxiomas[\paratodo{hab}{h} \paratodo{conj(posicion)}{ps} \paratodo{posicion}{p} \paratodo{nat}{n,k,tam}]
  \tadAlinearAxiomas{estaOcupada?(ocupar(h,p'),p)}
    \tadAxioma{esValida?(nueva(n),p)}{0 $\leq$ $\Pi_1$(p) < \ n $\land$ 0 $\leq$ $\Pi_2$(p) < \ n}
    
    \tadAxioma{esValida?(ocupar(h,p'),p)}{p = p' \oluego \ esValida?(h, p)}
    
    \tadAxioma{estaOcupada?(nueva(n),p)}{false}
    
    \tadAxioma{estaOcupada?(ocupar(h,p'),p)}{p = p' $\lor$ estaOcupada?(h, p)}
    
    \tadAxioma{tamano(nueva(n))}{n}
    
    \tadAxioma{tamano(ocupar(h, p))}{tamano(h)}
    
    \tadAxioma{esConexa?(h)}{verificarAlcance(h, posicionesLibres(posiciones(h)))}

   \tadAxioma{posicionesLibres(h, ps)}{\textbf{if} \ $\emptyset?(ps)$ \\ \textbf{then} \ $\emptyset$ \\ \textbf{else} \\ (\textbf{if} estaOcupada?(h, dameUno(ps)) \textbf{then} \ $\emptyset$ \textbf{else} \{dameUno(ps)\} \textbf{fi}) \\ $\cup$ posicionesLibres(h, sinUno(ps)) \\ \textbf{fi}}

    \tadAxioma{verificarAlcance(h, ps)}{\textbf{if} \ $\emptyset?(ps)$ \\ \textbf{then} \ true \\ \textbf{else} \\ verificarAlancePos(h, ps, dameUno(ps)) $\land$ verificarAlcance(h, p) \\ \textbf{fi}}
    
    \tadAxioma{verificarAlcancePos(h, ps, p)}{\textbf{if} \ $\emptyset?(ps)$ \\ \textbf{then} \ true \\ \textbf{else} \\ esAlcanzable(h, p, dameUno(ps)) $\land$ verificarAlcancePos(h, p, sinUno(ps)) \\ \textbf{fi}}

    \tadAxioma{posiciones(h)}{darPosiciones(h, tamano(h) - 1, tamano(h) - 1, tamano(h) - 1)}

    \tadAxioma{darPosiciones(h, n, k, tam)}{\textbf{if} \ $n = 0? \land k = 0?$ \\ \textbf{then} \ $\emptyset$ \\ \textbf{else if} $k = 0?$ \\ \textbf{then} \ Ag((n,k), darPosiciones(h, n - 1, tam, tam)) \\ \textbf{else} \ Ag((n,k), darPosiciones(h, n, k - 1, tam))  \\ \textbf{fi} \\ \textbf{fi}}

\end{tad}

\newpage

\section{TAD \tadNombre{Accion}}
\begin{tad}{\tadNombre{Accion}}
    \tadGeneros{accion}
    
    \tadExporta{accion, observadores, generadores, genero, otras operaciones}
    \tadUsa{\tadNombre{Direccion, Posicion, Ubicacion, Bool, Conjunto, Habitacion, Secuencia}}
    \tadIgualdadObservacional{a}{a'}{accion}{
        esNada(a) $\igobs$ esNada(a') $\land$ \\
        esDisparar(a) $\igobs$ esDisparar(a') $\land$ \\
        esMover(a) $\igobs$ esMover(a') $\land$ \\
        esMirar(a) $\igobs$ esMirar(a') $\land$ \\
        ((esMover(a) $\lor$ esMirar(a)) $\impluego$ \\ 
         direccion(a) $\igobs$ direccion(a'))
    }

    \tadAlinearFunciones{posicionesAfectadasPor}{hab/h,ubicacion/u,secu(accion)}{}{}
    
    \tadObservadores
        \tadOperacion{esMover}{accion}{bool}{}
        \tadOperacion{esMirar}{accion}{bool}{}
        \tadOperacion{esDisparar}{accion}{bool}{}
        \tadOperacion{esNada}{accion}{bool}{}
        \tadOperacion{direccion}{accion/a}{direccion}{esMirar(a) $\lor$ esMover(a)}

    \tadGeneradores
        \tadOperacion{mover}{direccion}{accion}{}
        \tadOperacion{mirar}{direccion}{accion}{}
        \tadOperacion{disparar}{}{accion}{}
        \tadOperacion{nada}{}{accion}{}

    \tadOtrasOperaciones
        \tadOperacion{ubicacionLuegoDe}{accion,hab/h,ubicacion/u}{conj(posicion)}{esValida?(h, pos(u))}
        \tadOperacion{posicionesAfectadasPor}{accion,hab/h,ubicacion/u}{conj(posicion)}{esValida?(h, pos(u))}
        \tadOperacion{$\neg$ \argumento}{accion}{accion}{}
        \tadOperacion{invertir}{hab/h,ubicacion/u,secu(accion)}{secu(accion)}{esValida?(h, pos(u))}

    \tadAlinearAxiomas{posicionesAfectadasPor(mover(d), h, u)}
    \tadAxiomas[
            \paratodo{direccion}{d}
            \paratodo{ubicacion}{u}, 
            \paratodo{habitacion}{a},
            \paratodo{secu(accion)}{as}
        ]
        \tadAxioma{posicionesAfectadasPor(mover(d), h, u)}{$\emptyset$}
        \tadAxioma{posicionesAfectadasPor(mirar(d), h, u)}{$\emptyset$}
        \tadAxioma{posicionesAfectadasPor(nada, h, u)}{$\emptyset$}
        \tadAxioma{posicionesAfectadasPor(disparar, h, u)}{
            \ifCustom
            {esValida?(h, proxPosEnDir(dir(u), pos(u)) $\yluego$ \\
             $\neg$ estaOcupada?(h, proxPosEnDir(dir(u), pos(u)))}
            {Ag(proxPosEnDir(dir(u), pos(u)), \\
                posicionesAfectadasPor(disparar, h, \\
                                      \tupla{proxPosEnDir(dir(u), pos(u)), dir(u)}))}
            {$\emptyset$}
        }

        \tadAxioma{invertir(h, u, as)}{
            \ifCustom
                {vacia?(as)}
                {\secuencia{}}
                {\\
                 invertir(h, ubicacionLuegoDe(prim(as), h, u), fin(as)) \circulito \\
                 $\neg$(prim(as), h, u)}
        }
        \tadAxioma{$\neg$(mover(d), h, u)}{
            \textbf{if} pos(ubicacionLuegoDe(mover(d), h, u)) = pos(u) \\
            \textbf{then} mirar(opuesta(d)) \\
            \textbf{else} mover(opuesta(d)) \\
            \textbf{fi}
        }
        \tadAxioma{$\neg$(mirar(d), h, u)}{mirar(opuesta(d))}
        \tadAxioma{$\neg$(disparar, h, u)}{disparar}
        \tadAxioma{$\neg$(nada, h, u)}{nada}
        \tadAxioma{ubicacionLuegoDe(nada, h, u)}{u}
        \tadAxioma{ubicacionLuegoDe(disparar, h, u)}{u}
        \tadAxioma{ubicacionLuegoDe(mirar(d), h, u)}{\tupla{pos(u), d}}
        \tadAxioma{ubicacionLuegoDe(mover(d), h, u)}{
            \tupla{
                (\ifCustom
                    {esValida?(h, proxPosEnDir(d, pos(u))) $\yluego$ \\
                     $\neg$estaOcupada?(h, proxPosEnDir(d, pos(u)))}
                    {proxPosEnDir(d, pos(u))}
                    {pos(u)}), 
                d
            }
        }
        \tadAxioma{esMirar(mirar(d))}{\textbf{true}}
        \tadAxioma{esMirar(mover(d))}{false}
        \tadAxioma{esMirar(disparar)}{false}
        \tadAxioma{esMirar(nada)}{false}
        \tadAxioma{esMover(mirar(d))}{false}
        \tadAxioma{esMover(mover(d))}{\textbf{true}}
        \tadAxioma{esMover(disparar)}{false}
        \tadAxioma{esMover(nada)}{false}
        \tadAxioma{esDisparar(mirar(d))}{false}
        \tadAxioma{esDisparar(mover(d))}{false}
        \tadAxioma{esDisparar(disparar)}{\textbf{true}}
        \tadAxioma{esDisparar(nada)}{false}
        \tadAxioma{esNada(mirar(d))}{false}
        \tadAxioma{esNada(mover(d))}{false}
        \tadAxioma{esNada(disparar)}{false}
        \tadAxioma{esNada(nada)}{\textbf{true}}
        \tadAxioma{direccion(mirar(d))}{d}
        \tadAxioma{direccion(mover(d))}{d}
\end{tad}

\section{TAD \tadNombre{Direccion}}
\begin{tad}{\tadNombre{Direccion}}
    \tadGeneros{direccion}
    \tadExporta{direccion, observadores, generadores, otras operaciones}
    \tadUsa{\tadNombre{Bool, Posicion, Nat}}
    \tadIgualdadObservacional{d}{d'}{direccion}{
        esArriba(d) $\igobs$ esArriba(d') $\land$ \\
        esAbajo(d) $\igobs$ esAbajo(d') $\land$ \\
        esIzquierda(d) $\igobs$ esIzquierda(d') $\land$ \\ 
        esDerecha(d) $\igobs$ esDerecha(d')
    }
    \tadAlinearFunciones{proxPosEnDir}{direccion, posicion}{}
    \tadObservadores
        \tadOperacion{esArriba}{direccion}{bool}{}
        \tadOperacion{esAbajo}{direccion}{bool}{}
        \tadOperacion{esIzquierda}{direccion}{bool}{}
        \tadOperacion{esDerecha}{direccion}{bool}{}

    \tadGeneradores
        \tadOperacion{arriba}{}{direccion}{}
        \tadOperacion{abajo}{}{direccion}{}
        \tadOperacion{izquierda}{}{direccion}{}
        \tadOperacion{derecha}{}{direccion}{}
    
    \tadOtrasOperaciones
        \tadOperacion{opuesta}{direccion}{direccion}{}
        \tadOperacion{proxPosEnDir}{direccion, posicion}{posicion}{}

    \tadAlinearAxiomas{proxPosEnDir(izquierda, p)}
    \tadAxiomas[\paratodo{posicion}{p}]
        \tadAxioma{opuesta(arriba)}{abajo}
        \tadAxioma{opuesta(abajo)}{arriba}
        \tadAxioma{opuesta(izquierda)}{derecha}
        \tadAxioma{opuesta(derecha)}{izquierda}
        \tadAxioma{proxPosEnDir(arriba, p)}{\tupla{$\Pi_1$(p), $\Pi_2$(p) + 1}}
        \tadAxioma{proxPosEnDir(abajo, p)}{\tupla{$\Pi_1$(p), $\Pi_2$(p) - 1}}
        \tadAxioma{proxPosEnDir(izquierda, p)}{\tupla{$\Pi_1$(p) - 1, $\Pi_2$(p)}}
        \tadAxioma{proxPosEnDir(derecha, p)}{\tupla{$\Pi_1$(p) + 1, $\Pi_2$(p}}
        \tadAxioma{esArriba(arriba)}{\textbf{true}}
        \tadAxioma{esArriba(abajo)}{false}
        \tadAxioma{esArriba(izquierda)}{false}
        \tadAxioma{esArriba(derecha)}{false}
        \tadAxioma{esAbajo(arriba)}{false}
        \tadAxioma{esAbajo(abajo)}{\textbf{true}}
        \tadAxioma{esAbajo(izquierda)}{false}
        \tadAxioma{esAbajo(derecha)}{false}
        \tadAxioma{esIzquierda(arriba)}{false}
        \tadAxioma{esIzquierda(abajo)}{false}
        \tadAxioma{esIzquierda(izquierda)}{\textbf{true}}
        \tadAxioma{esIzquierda(derecha)}{false}
        \tadAxioma{esDerecha(arriba)}{false}
        \tadAxioma{esDerecha(abajo)}{false}
        \tadAxioma{esDerecha(izquierda)}{false}
        \tadAxioma{esDerecha(derecha)}{\textbf{true}}

\end{tad}{}

\section{Decisiones tomadas}

\begin{enumerate}
    \item Asumimos la existencia de las operaciones localizarJugadores y esAlcanzable, entonces no las axiomatizamos.
    \item Tomamos como validas dentro del juego habitaciones de 2x2 o mas, ya que sino nadie moriria.
    \item Como los nombres de los fantasmas son Nat, el nombre del nuevo es el cardinal del el conjunto de fantasmas \text{+ 1}. 
    Asi tambien se puede facilmente saber cual es el fantasma especial.

    \item La accion mirar fue agregada para considerar el caso patologico del fantasma.
    
    Esta fue reestringida para que solo se pueda usar en ese caso.
    
    Esto llevo a la creacion del TAD Direccion como un desprendimiento de Accion.
\end{enumerate}

\end{document}