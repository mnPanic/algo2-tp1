\documentclass[10pt, a4paper]{article}
\usepackage[paper=a4paper, left=1.5cm, right=1.5cm, bottom=1.5cm, top=3.5cm]{geometry}
\usepackage[latin1]{inputenc}
\usepackage[T1]{fontenc}
\usepackage[spanish]{babel}
\usepackage{indentfirst}
\usepackage{fancyhdr}
\usepackage{latexsym}
\usepackage{lastpage}
\usepackage{aed2-symb,aed2-itef,aed2-tad}
\usepackage[colorlinks=true, linkcolor=blue]{hyperref}
\usepackage{calc}

\newcommand{\f}[1]{\text{#1}}
\renewcommand{\paratodo}[2]{$\forall~#2$: #1}

\sloppy

\hypersetup{%
 % Para que el PDF se abra a p�gina completa.
 pdfstartview= {FitH \hypercalcbp{\paperheight-\topmargin-1in-\headheight}},
 pdfauthor={C�tedra de Algoritmos y Estructuras de Datos II - DC - UBA},
 pdfkeywords={TADs b�sicos},
 pdfsubject={Tipos abstractos de datos b�sicos}
}

\parskip=5pt % 10pt es el tama�o de fuente

% Pongo en 0 la distancia extra entre �temes.
\let\olditemize\itemize
\def\itemize{\olditemize\itemsep=0pt}

% Acomodo fancyhdr.
\pagestyle{fancy}
\thispagestyle{fancy}
\addtolength{\headheight}{1pt}
\lhead{Algoritmos y Estructuras de Datos II}
\rhead{$1^{\mathrm{er}}$ cuatrimestre de 2012}
\cfoot{\thepage /\pageref{LastPage}}
\renewcommand{\footrulewidth}{0.4pt}

\author{Algoritmos y Estructuras de Datos II, DC, UBA.}
\date{}
\title{Tipos abstractos de datos b�sicos}

\begin{document}
%Pagina de titulo e indice
\thispagestyle{empty}

\maketitle
\tableofcontents

\newpage

% TADs
\section{TAD \tadNombre{PJ}}
\section{TAD \tadNombre{Fantasma}}
\section{TAD \tadNombre{Juego}}

\begin{tad}{\tadNombre{Juego}}
    \tadGeneros{juego}
    \tadExporta{TODO}
    \tadUsa{\tadNombre{Habitacion}}

    \tadIgualdadObservacional{j}{j'}{juego}{$(n=0? \igobs m=0?)\ \yluego$\\ $(\neg(n=0?) \impluego (\text{pred}(n) \igobs \text{pred}(m)))$}
    \tadIgualdadObservacional{j}{j'}{juego}{
        (accionesPJs(j) $\igobs$ accionesPJs(j')) $\land$ \\
        (accionesFan(j) $\igobs$ accionesFan(j')) $\land$ \\
        (localizarJugadores(j) $\igobs$ localizarJugadores(j')) $\land$ \\
        (hab(j) $\igobs$ hab(j')) $\land$ \\
        (($\forall$ p : pj) (vivePJ?(j, p) \igobs vivePJ?(j', p))) $\land$ \\
        (($\forall$ f : fantasma) ((viveFan?(j, p) \igobs viveFan?(j', p)) $\land$ \\
                                   (ubicacionInicialFan(j, f) \igobs ubicacionInicialFan(j', f)))
    }

    
    \tadAlinearFunciones{ubicacionInicialFan}{juego/j, fantasma/f}

    \tadObservadores
        \tadOperacion{accionesPJs}{juego}{dicc(pj, secu(accion))}{}
        \tadOperacion{accionesFan}{juego}{dicc(pj, secu(accion))}{}
        \tadOperacion{hab}{juego}{hab}{}
        \tadOperacion{vivePJ?}{juego/j, pj/p}{bool}{p $\in$ jugadores(j)}
        \tadOperacion{viveFan?}{juego/j, fantasma/f}{bool}{f $\in$ fantasmas(j)}
        \tadOperacion{ubicacionInicialFan}{juego/j, fantasma/f}{ubicacion}{f $\in$ fantasmas(f)}
        \tadOperacion{localizarJugadores}{juego}{dicc(pj, ubicacion)}{}
    
    \tadGeneradores

    \tadOtrasOperaciones

    \tadAxiomas[\paratodo{nat}{n, m}]
    \tadAlinearAxiomas{pred(suc($n$))}
    \tadAxioma{$0=0?$}{true}

\end{tad}

\end{document}